\chapter*{Введение}
\addcontentsline{toc}{chapter}{Введение}

Расстояния Левенштейна и Дамерау-Левенштейна применяются в таких сферах, как: 
\begin{itemize}
	\item компьютерная лингвистика (автозамена в посиковых запросах, текстовая редактура);
	\item биоинформатика (последовательности белков);
	\item нечеткий поиск записей в базах (борьба с мошенниками и опечатками);
	\item сравнения текстовых файлов утилитой \code{diff} и ей подобными (здесь роль «символов» играют строки, а роль «строк» — файлы);
	\item сравнения генов, хромосом и белков в биоинформатике.
\end{itemize}


Задачами данной лабораторной являются:

\begin{enumerate}
	\item построение алгоритмов Левенштейна и Дамерау-Левенштейна;
	\item сравнение времени выполнения различный версий алгоритма Дамерау-Левенштейна, а именно нерекурсивного, рекурсивного, рекурсивного с кешированием;
	\item составление отчета о проделанной работе.
\end{enumerate}

Целью настойщей лабораторной работы является исследование быстродействия алгоритмов Левенштейна и Дамерау-Левенштейна.
