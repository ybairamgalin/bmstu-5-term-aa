\chapter{Исследовательская часть}

В данном разделе будет приведен равнительный анализ алгоритмов на основе
полученных данных.

\section{Технические характеристики}

Технические характеристики устройства, на котором выполнялось тестирование,
следующие.

\begin{itemize}
	\item Операционная система: macOs Monterey 12.4\cite{ubuntu}.
	\item Память: 16 Гбайт.
	\item Процессор: 2,6 ГГц 6‑ядерный процессор Intel Core i7\cite{intel}.
\end{itemize}

Тестирование проводилось на ноутбуке, включенном в сеть электропитания.
Во время тестирования ноутбук был нагружен только встроенными приложениями
окружения, а также непосредственно системой тестирования.


\section{Время выполнения алгоритмов}

В таблице \ref{tab:time0} представлены замеры времени работы алгоритмов. Здесь и далее: М — матричный алгоритм, Р — рекурсивный алгоритм, РК -- рекурсивный алгоритм с кешем. Время в микросекундах. Прочерк <<—>>  означает, что время выполнения алгоритма слишком велико.

\begin{table}[h]
	\begin{center}
		\captionsetup{justification=raggedright,singlelinecheck=off}
		\caption{\label{tab:time0}Результаты замеров времени алгоритмов (миллисекунды)}
		\begin{tabular}{|c|c|c|c|c|}
		\hline
		Строка 1 & Строка 2 &  М & Р & РК \\
		\hline
		Similar  & Similar & 19 & 1075 & 23\\
		\hline
		kjFFkjRhtZ  & ZFdbYi4nQd & 21 & 17042 & 59\\
		\hline
		words diff  & wrods dfif & 20 & 139389 & 52\\
		\hline
		Two similar sentences  & Two similar sentences & 57 & — & 113 \\
		\hline
		NVyhNfPnYykGwDZETiI5  & GD34eG0rIZ73qGorddY5 & 50 &  — & 94\\
		\hline
		some real words with errors  & smoe real wrods wiht erors & 84 & — & 66\\
		\hline
		\end{tabular}
	\end{center}
\end{table}

\newpage

\section*{Вывод}
При увеличении размера матриц, увеличивается и эффективность работы алгоритма Винограда в сравнении со стандартным алгоритмом. В среднем реализация по Винограду работает быстрее в 1.7 раз. Оптимизированная реализация алгоритма Винограда также позволяет уменьшить время работы при больших размерностях: например, для размерности матрицы, равной 200, эксперимент показал, что алгоритм Винограда быстрее стандартного алгоритма чуть менее, чем в 2 раза, когда оптимизированная версия выигрывает у обычного умножения более, чем в 2 раза. 
