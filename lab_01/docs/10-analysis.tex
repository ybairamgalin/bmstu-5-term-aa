\chapter{Аналитическая часть}

Расстоянием Левенштейна для двух строк $a, b$ называют число d, такое что

\begin{equation}
	\label{eq:D}
	d(|a|, |b|) = \begin{cases}
		
		0, &\text{если i = 0, j = 0;}\\
		i, &\text{если j = 0, i > 0;}\\
		j, &\text{если i = 0, j > 0;}\\
		\min  (d(i-1, j) + 1 ,\\
		\qquad d(i, j-1) + 1 ,&\text{иначе.}\\
		\qquad d(i-1, j-1) + X(a[i], b[j]) \\
		)
	\end{cases}
\end{equation}
где $|m|$ — длина строки $m$; X(a[i], b[j]) — равно 0, если $a[i] = b[j]$, 0 иначе.

В формуле \ref{eq:D} выражение
\begin{equation}
	\label{eq:delete}
	d(i-1, j) + 1
\end{equation}
соответствует операции удаления $j$-го символа из строки $a$; выражение

\begin{equation}
	\label{eq:delete}
	d(i, j-1) + 1
\end{equation}
соответствует операции добавления $i$-го символа к строке $a$; выражение

\begin{equation}
	\label{eq:delete}
	d(i-1, j-1) + X(a[i], b[j])
\end{equation}
соответствует операцие замены $i$-го символа строки $a$ на $j$-ый символ строки $b$, если такая замена требуется (символы различны).


\pagebreak
Расстоянием Дамерау-Левенштейна для двух строк $a, b$ называют число d, такое что
\begin{equation}
	\label{eq:d}
	d_{a,b}(i, j) = \begin{cases}
		\max(i, j), &\text{если }\min(i, j) = 0;\\
		\min ( \\
		\qquad d_{a,b}(i, j-1) + 1,&\text{если i, j > 0}\\
		\qquad d_{a,b}(i-1, j) + 1, &\text{и $a[i]=b[j-1]$}\\
		\qquad d_{a,b}(i-1, j-1) + X(a[i], b[j]), &\text{и a$[i-1]=b[j]$;}\\
		\qquad d_{a,b}(i-2, j-2) + 1 \\
		) \\
		\min (\\
		\qquad d_{a,b}(i, j-1) + 1,\\
		\qquad d_{a,b}(i-1, j) + 1,&\text{иначе.}\\
		\qquad d_{a,b}(i-1, j-1) + X(a[i], b[j]) \\
		) \\
	\end{cases},
\end{equation}

\section{Рекурсивный алгоритм Дамерау — Левенштейна}


Рекурсивный алгоритм по своей сути повторяет определение из формулы \ref{eq:d}. Он является крайне неэффективным, и работает неприемлимо долго даже на относительно небольших строках (12-15) символов.


\section{Рекурсивный алгоритм Дамерау — Левенштейна с кешированием}

Идея использования кеширования в рекурсивном влгоритме Дамерау-Левенштейна состоит в том, чтобы запоминать уже подсчитанные значения $d(i, j)$. Для этого необходимо создать матрицу $M$ размера $i, j$, заполнить ее, например, значение -1. После вычисления очередного $d(i, j)$ стоит поместить его значение в  $M_ij$.


\section{Матричный алгоритм нахождения расстояния Дамерау — Левенштейна}

При больших $i, j$ прямая реализация формулы \ref{eq:D} может быть малоэффективна по времени исполнения, так как множество промежуточных значения $ D(i, j)$ вычисляются не по одному разу. Для оптимизации нахождения можно использовать матрицу для хранения соответствующих промежуточных значений.

Матрица размером $(length(S1)+ 1)$x$((length(S2) + 1)$, где $length(S)$ — длина строки S. Значение в ячейке $[i, j]$ равно значению $D(S1[1...i], S2[1...j])$. Первая строка и первый столбец тривиальны. 

Всю таблицу (за исключением первого столбца и первой строки) заполняем в соответствии с формулой \ref{eq:mat}.
\begin{equation}
	\label{eq:mat}
	A[i][j] = min \begin{cases}
		A[i-1][j] + 1\\
		A[i][j-1] + 1\\
		A[i-1][j-1] + X(S1[i], S2[j])
	\end{cases}.
\end{equation}

В результате расстоянием Левенштейна будет ячейка матрицы с индексами $i = length(S1$) и $j = length(S2)$.


\section*{Вывод}

Рассмотрены теоретические аспекты алгоритмов Левенштейна и Дамерау — Левештейна. Получены достаточные знания для программной реализации.

\clearpage