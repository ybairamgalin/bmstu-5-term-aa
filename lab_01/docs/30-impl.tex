\chapter{Технологическая часть}

В данном разделе будут приведены требования к программному обеспечению, используемые технологии и реализации алгоритмов.

\section{Требования к программному обеспечению}

К программе предъявляется ряд требований:

\begin{itemize}
	\item на вход подаются 2 произвольные строки ASCII символов;
	\item в случае успешного выполнения возвращается одно целое число — расстояние Дамерау-Левенштейна;
	\item в случае если по какой-либо причине не удалось найти расстояние Дамерау-Левенштейна — вывести ошибку.
\end{itemize}

\section{Выбор средств реализации}

В качестве языка программирования для реализации данной лабораторной работы был выбран язые программирования C++\cite{pythonlang}.

Язык позволяет управлять всеми ресурсами компьютера и, тем самым позволяет
писать эффективные алгоритмы.

Время работы алгоритмов было замерено с помощью функции из листинга \ref{lst:exec_time}, разработанной самостоятельно.

\clearpage
\begin{lstlisting}[label=lst:exec_time, caption=Функция для замера времени исполнения функции]
	template <typename F, typename... Args>
	auto GetExecutionTime(F function, Args&&... args) {
		const auto start_time = std::chrono::high_resolution_clock::now();
		const auto return_value = function(std::forward<Args>(args)...);
		const auto end_time = std::chrono::high_resolution_clock::now();
		return std::pair(
				std::chrono::duration_cast<TimeUnit>(end_time - start_time),
				return_value);
	}
\end{lstlisting}

\section{Реализация алгоритмов}

В листингах 3.2, 3.3, 3.4 представлены реализации матричного алгоритма Дамерау-Левенштейна, 
рекурсивного алгоритма Дамерау-Левенштейна, рекурсивного алгоритма Дамерау-Левенштейна с кешированием.

\begin{lstlisting}[caption=Матричный алгоритм Дамерау-Левенштейна]
int DamerauLevenshteinDistance(
    const std::string& first, const std::string& second) {
	using Row = std::vector<int>;
	using Matrix = std::vector<Row>;
	
	Matrix distances(first.size() + 1, Row(second.size() + 1));
	
	for (size_t i = 0; i < distances.size(); ++i) {
		distances[i][0] = int(i);
	}
	for (size_t j = 0; j < distances[0].size(); ++j) {
		distances[0][j] = int(j);
	}
	for (size_t i = 1; i < distances.size(); ++i) {
		for (size_t j = 1; j < distances[0].size(); ++j) {
			int cost = 1;
			if (first[i - 1] == second[i - 1]) {
				cost = 0;
			}
			distances[i][j] = std::min({
				distances[i - 1][j] + 1,
				distances[i][j - 1] + 1,
				distances[i - 1][j - 1] + cost});
			
			if (i > 1 && j > 1 &&
			first[i - 1] == second[j - 2] && first[i - 2] == second[j - 1]) {
				distances[i][j] = std::min({
					distances[i][j], distances[i - 2][j - 2] + cost});
			}
		}
	}
	
	return distances[first.size()][second.size()];
}
\end{lstlisting}

\begin{lstlisting}[caption=Рекурсивный алгоритм Дамерау-Левенштейна]
int DamerauLevenshteinDistanceRecursive(
const std::string& first, const std::string& second) {
	if (first.empty()) {
		return int(second.size());
	}
	if (second.empty()) {
		return int(first.size());
	}
	
	std::function<int(int, int)> calculate_distance_recursively;
	calculate_distance_recursively = [&](int i, int j) -> int {
		if (i == 0 && j == 0) {
			return (first[i] != second[j]) ? 1 : 0;
		}
		if (i == 0) {
			return int(j);
		}
		if (j == 0) {
			return int(i);
		}
		
		int cost = (first[i] != second[j]) ? 1 : 0;
		
		int delete_operation_cost = calculate_distance_recursively(i - 1, j) + 1;
		int insert_operation_cost = calculate_distance_recursively(i, j - 1) + 1;
		int correspondence_operation_cost =
		calculate_distance_recursively(i - 1, j - 1) + cost;
		if (i > 1 && j > 1 &&
		first[i] == second[j - 1] && first[i - 1] == second[j]) {
			int swap_operation_cost =
			calculate_distance_recursively(i - 2, j - 2) + 1;
			return std::min({
				delete_operation_cost, insert_operation_cost,
				correspondence_operation_cost, swap_operation_cost});
		}
		
		return std::min({
			delete_operation_cost, insert_operation_cost,
			correspondence_operation_cost});
	};
	
	return calculate_distance_recursively(
	int(first.size() - 1), int(second.size() - 1));
}
\end{lstlisting}

\begin{lstlisting}[caption=Рекурсивный алгоритм Дамерау-Левенштейна с кешированием]
int DamerauLevenshteinDistanceRecursiveCached(
const std::string& first, const std::string& second) {
	using Row = std::vector<std::optional<int>>;
	using Matrix = std::vector<Row>;
	
	if (first.empty()) {
		return int(second.size());
	}
	if (second.empty()) {
		return int(first.size());
	}
	
	Matrix calculated_distances(first.size() + 1, Row(second.size() + 1));
	std::function<int(int, int)> calculate_distance_recursively;
	calculate_distance_recursively = [&](int i, int j) -> int {
		if (i == 0 && j == 0) {
			return (first[i] != second[j]) ? 1 : 0;
		}
		if (i == 0) {
			return int(j);
		}
		if (j == 0) {
			return int(i);
		}
		if (calculated_distances[i][j].has_value()) {
			return calculated_distances[i][j].value();
		}
		
		int cost = (first[i] != second[j]) ? 1 : 0;
		
		int delete_operation_cost = calculate_distance_recursively(i - 1, j) + 1;
		int insert_operation_cost = calculate_distance_recursively(i, j - 1) + 1;
		int correspondence_operation_cost =
		calculate_distance_recursively(i - 1, j - 1) + cost;
		if (i > 1 && j > 1 &&
		first[i] == second[j - 1] && first[i - 1] == second[j]) {
			int swap_operation_cost =
			calculate_distance_recursively(i - 2, j - 2) + 1;
			int result = std::min({
				delete_operation_cost, insert_operation_cost,
				correspondence_operation_cost, swap_operation_cost});
			calculated_distances[i][j] = result;
			return result;
		}
		
		int result = std::min({
			delete_operation_cost, insert_operation_cost,
			correspondence_operation_cost});
		calculated_distances[i][j] = result;
		
		return result;
	};
	
	return calculate_distance_recursively(
	int(first.size() - 1), int(second.size() - 1));
}
\end{lstlisting}


\section{Тестрование}



В таблице \ref{tab:tests} рассмотрены случаи ожидаемого поведения программы. Все тесты были успешно пройдены.

\begin{table}[h]
	\begin{center}
		\captionsetup{justification=raggedright,singlelinecheck=off}
		\caption{\label{tab:tests} Функциональные тесты}
		\begin{tabular}{|c|c|c|c|c|}
		\hline
		Строка 1 & Строка 2 & Ожидаемый результат \\ 
		\hline
		<<пустая строка>> & <<пустая строка>> & 0 \\ 
		\hline
		cat & hat & 1 \\ 
		\hline
		apple & aplpe & 1 \\ 
		\hline
		qwerty & queue & 4 \\ 
		\hline
		wolf & wolf & 0 \\
		\hline
		word &  <<пустая строка>>& 4 \\
		\hline
		mitsubishi & mercedes-benz & 11 \\
		\hline
		

		\end{tabular}
	\end{center}
\end{table}


\newpage
\section*{Вывод}

Разработаны и тестированы 3 алгоритма. Все алгоритмы соответствуют заявленным требованиям.
