\chapter{Аналитическая часть}
В этом разделе представлены описания алгоритмов порязрядной сортировки,
сортировки бинарным деревом и сортировки бусинами.

\section{Поразрядная сортировка}

\textbf{Поразрядная сортировка}\cite{sheyker} — один из немногих алгоритмов,
выполняющих сортировку за линейное (относительно количества элементов в
коллекции) время.


Исходно предназначен для сортировки целых чисел, записанных цифрами.
Однако, в памяти компьютеров любая информация записывается целыми числами,
алгоритм пригоден для сортировки любых объектов, запись которых можно поделить
на «разряды», содержащие сравнимые значения. Например, так сортировать можно
не только числа, записанные в виде набора цифр, но и строки, являющиеся набором
символов, и вообще произвольные значения в памяти, представленные в виде
набора байт.


Сравнение производится поразрядно: сначала сравниваются значения одного
крайнего разряда, и элементы группируются по результатам этого сравнения,
затем сравниваются значения следующего разряда, соседнего, и элементы либо
упорядочиваются по результатам сравнения значений этого разряда внутри
образованных на предыдущем проходе групп, либо переупорядочиваются в
целом, но сохраняя относительный порядок, достигнутый при предыдущей
сортировке. Затем аналогично делается для следующего разряда, и так до конца.


\section{Сортировка бинарным деревом}

\textbf{Сортировка бинарным деревом}\cite{insert} — алгоритм сортировки,
при котором над коллекцией входных данных строится один из видов бинарного
дерева (например, красно-черное дерево или дерево двоичного поиска). Так как
одно из свойств дерева поиска - отсортированной в любой момент времени, после
построения дерева уже отсортированные данные можно переместить в исходную
коллекцию любого вида.


Алгоритм является одним из самых эффективных
способов сортировки при чтении данных из потока (файла, сокета, консоли).


Стоит отметить, что худший случай достигается при уже отсортированных (в прямом
или обратном порядке) данных, поступающих на вход.

\section{Сортировка бусинами}

\textbf{Сортировка бусинами}\cite{select} - алгоритм сортировки, который
применим только для целых чисел (или данных которые могут быть однозначно
отображены в целые числа). Каждое целое число $n$ в таком случае представлется в
виде набора бусин длины $n$ и надевается на шесты (количество шестов должно
быть равно максимальному значению коллекции). Затем бусины начинают скользить
и на ряде $i$ оказывается количетсво бусин, которое соответствует
значению $i$-го элемента отсотритованной коллекции данных. На мой взгляд,
наиболее иллюстративными для понимания алгоритма являются рисунки
\ref{img:bead_sort_1},~\ref{img:bead_sort_2}.

\img{80mm}{bead_sort_1}{Сортировка бусинами - изначальное положение}
\img{80mm}{bead_sort_2}{Сортировка бусинами - отсортированная коллекция}


\section*{Вывод}

Изучены концептуальные особенности выбранных алгоритмоы сортировки. Получено
достаточное понимание идей работы алгоритмов для их реализации.
