\chapter*{Заключение}
\addcontentsline{toc}{chapter}{Заключение}

Алгоритм сортировки бусинами является крайне неэффективным и может представлять интерес лишь в учебных целях.

Алгоритм сортировки бинарным деревом (при использовании обычного ДДП) деградирует до сложности $O(n^2)$ на отсортированных данных. Из этого можно сделать вывод, что при наличии  вероятности появления отсортированных (или близких к таковым) входных данных лучше использовать какой-либо вид сбалансированных деревьев. Таковыми могут быть АВЛ, красно-черное дерево и некоторые другие.

Алгоритм побитовой сортировки является  эффективным способом сортироки данных, каждый элемент которых, имеет небольшой размер.