\chapter*{Введение}
\addcontentsline{toc}{chapter}{Введение}

Сортировка являдется одной из наиболее часто используемых процедур обработки
информации.

Обычно говорят о возможности сортировки некоторого множества или мультимножества
элементов, если на таком множестве:
\begin{itemize}
    \item для каждых двух элементов задано отношение порядка;
    \item выполнен закон трихотомии;
    \item выполнен закон транзитивности.
\end{itemize}

Таким образом задача сортировки будет заключаться в нахождении такого отображения
мультимножества $X$ в кортеж $(x_1, x_2, \dots, x_n)$,
что для любого $x_i : i \in [1, N] : x_i < x_{i+1}$.


Одной из наиболее часто используемых характеристик, используемых для сравнения
эффективности алгоритмов является асимптотическая временная сложность. Данная
характеристика имеет особую важность применительно к алгритмам сортировки, так
как на практике размер соритруемых данных может быть велик и
использование асиптотически неподходящего алгоритма сортировки может привести к
неэффективному использованию ресурсов.


Цели данной лабораторной:

\begin{itemize}
    \item изучить и реализовать три алгоритма сортировки: поразрядная, бинарным деревом, бусинами;
    \item провести сравнительный анализ времени выполнения алгоритмов на основе теоретических вычислений;
    \item провести сравнительный анализ времени выполнения алгоритмов на основе экспериментальных данных;
    \item описать полученные результаты в отчете о выполнении лабораторной работы.
\end{itemize}
