\chapter{Технологическая часть}

В данном разделе будут приведены требования к программному обеспечению, используемые технологии и реализации алгоритмов.

\section{Требования к программному обеспечению}

К программе предъявляется ряд требований:

\begin{itemize}
	\item на вход подаются 2 произвольные матрицы с численным типом элементов;
	\item при невозмодности выполнить умножение кидается исключение;
	\item в случае успешного выполнения программы на выходе матрица, которая является результатом умножения первой матрицы на вторую (в порядке их поступления на вход).
\end{itemize}

\section{Выбор средств реализации}

В качестве языка программирования для реализации данной лабораторной работы был выбран язые программирования C++\cite{pythonlang}.

Язык позволяет управлять всеми ресурсами компьютера и, тем самым позволяет
писать эффективные алгоритмы.

Время работы алгоритмов было замерено с помощью функции из листинга \ref{lst:exec_time}, разработанной самостоятельно.

\clearpage
\begin{lstlisting}[label=lst:exec_time, caption=Функция для замера времени исполнения функции]
	template <typename F, typename... Args>
	auto GetExecutionTime(F function, Args&&... args) {
		const auto start_time = std::chrono::high_resolution_clock::now();
		const auto return_value = function(std::forward<Args>(args)...);
		const auto end_time = std::chrono::high_resolution_clock::now();
		return std::pair(
				std::chrono::duration_cast<TimeUnit>(end_time - start_time),
				return_value);
	}
\end{lstlisting}

\section{Реализация алгоритмов}

В листингах 3.1, 3.2, 3.3 представлены реализации алгоритма обычного умножения матриц, алгоритма Винограда, алгоритма Винограда с оптимизациями соответственно.

\begin{lstlisting}[caption=Функция стандартного умножения матриц]
	template<typename ValueType>
	models::Matrix<ValueType> MultiplySimple(
	const models::Matrix<ValueType>& first,
	const models::Matrix<ValueType>& second) {
		if (first.GetSize().second != second.GetSize().first) {
			throw std::runtime_error("Inappropriate matrix dimensions to execute multiplication");
		}
		
		const auto result_size = std::pair(first.GetSize().first,
		second.GetSize().second);
		models::Matrix<ValueType> result{result_size.first, result_size.second};
		for (std::size_t i = 0; i < result_size.first; ++i) {
			for (std::size_t j = 0; j < result_size.second; ++j) {
				for (std::size_t k = 0; k < first.GetSize().second; ++k) {
					result[i][j] += first[i][k] * second[k][j];
				}
			}
		}
		
		return result;
	}
\end{lstlisting}

\begin{lstlisting}[caption=Функция алгоритма Винограда умножения матриц]
	template<typename ValueType>
	models::Matrix<ValueType> MultiplyVinograd(
	const models::Matrix<ValueType>& first,
	const models::Matrix<ValueType>& second) {
		if (first.GetSize().second != second.GetSize().first) {
			throw std::runtime_error("Inappropriate matrix dimensions to execute multiplication");
		}
		
		typename models::Matrix<ValueType>::Row row_factors(first.GetSize().first);
		for (std::size_t i = 0; i < first.GetSize().first; ++i) {
			for (std::size_t j = 0; j < first.GetSize().second - 1; j = j + 2) {
				row_factors[i] = row_factors[i] + first[i][j] * first[i][j + 1];
			}
		}
		typename models::Matrix<ValueType>::Row
		column_factors(second.GetSize().second);
		for (std::size_t i = 0; i < second.GetSize().second; ++i) {
			for (std::size_t j = 0; j < second.GetSize().first - 1; j = j + 2) {
				column_factors[i] = column_factors[i] + second[j][i] * second[j + 1][i];
			}
		}
		
		const auto result_size = std::pair(first.GetSize().first,
		second.GetSize().second);
		models::Matrix<ValueType> result{result_size.first, result_size.second};
		for (std::size_t i = 0; i < result_size.first; ++i) {
			for (std::size_t j = 0; j < result_size.second; ++j) {
				result[i][j] = -row_factors[i] - column_factors[j];
				for (std::size_t k = 0; k < first.GetSize().second - 1; k = k + 2) {
					result[i][j] = result[i][j] + (first[i][k + 1] + second[k][j]) *
					(first[i][k] + second[k + 1][j]);
				}
			}
		}
		
		if (first.GetSize().second % 2 == 1) {
			for (std::size_t i = 0; i < result_size.first; ++i) {
				for (std::size_t j = 0; j < result_size.second; ++j) {
					result[i][j] =
					result[i][j] + first[i][first.GetSize().second - 1] *
					second[second.GetSize().first - 1][j];
				}
			}
		}
		
		return result;
	}
\end{lstlisting}

\begin{lstlisting}[caption=Функция оптимизированного алгоритма Винограда умножения матриц]
	template<typename ValueType>
	models::Matrix<ValueType> MultiplyVinogradOptimized(
	const models::Matrix<ValueType>& first,
	const models::Matrix<ValueType>& second) {
		if (first.GetSize().second != second.GetSize().first) {
			throw std::runtime_error("Inappropriate matrix dimensions to execute "
			"multiplication");
		}
		
		typename models::Matrix<ValueType>::Row row_factors(first.GetSize().first);
		for (std::size_t i = 0; i < first.GetSize().first; ++i) {
			const auto upper_index = first.GetSize().second - 1;
			for (std::size_t j = 0; j < upper_index; j += 2) {
				row_factors[i] += first[i][j] * first[i][j + 1];
			}
		}
		typename models::Matrix<ValueType>::Row
		column_factors(second.GetSize().second);
		for (std::size_t i = 0; i < second.GetSize().second; ++i) {
			const auto upper_index = second.GetSize().first - 1;
			for (std::size_t j = 0; j < upper_index; j += 2) {
				column_factors[i] += second[j][i] * second[j + 1][i];
			}
		}
		
		const auto result_size = std::pair(first.GetSize().first,
		second.GetSize().second);
		models::Matrix<ValueType> result{result_size.first, result_size.second};
		for (std::size_t i = 0; i < result_size.first; ++i) {
			for (std::size_t j = 0; j < result_size.second; ++j) {
				result[i][j] = -row_factors[i] - column_factors[j];
				const auto upper_index = first.GetSize().second - 1;
				for (std::size_t k = 0; k < upper_index; k += 2) {
					result[i][j] += (first[i][k + 1] + second[k][j]) *
					(first[i][k] + second[k + 1][j]);
				}
			}
		}
		
		if (first.GetSize().second % 2 == 1) {
			for (std::size_t i = 0; i < result_size.first; ++i) {
				for (std::size_t j = 0; j < result_size.second; ++j) {
					result[i][j] += first[i][first.GetSize().second - 1] *
					second[second.GetSize().first - 1][j];
				}
			}
		}
		
		return result;
	}
\end{lstlisting}


\section{Тестрование}



В таблице \ref{tab:tests} рассмотрены случаи ожидаемого поведения программы. Все тесты были успешно пройдены.

\begin{table}[h]
	\begin{center}
		\captionsetup{justification=raggedright,singlelinecheck=off}
		\caption{\label{tab:tests} Функциональные тесты}
		\begin{tabular}{|c|c|c|c|c|}
		\hline
		Матрица 1 & Матрица 2 & Ожидаемый результат \\ 
		\hline

		$\begin{pmatrix}
			&
		\end{pmatrix}$ &
		$\begin{pmatrix}
			&
		\end{pmatrix}$ &
		Сообщение об ошибке \\ \hline
			
		$\begin{pmatrix}
			1 & 5 & 7\\
			2 & 6 & 8\\
			3 & 7 & 9
		\end{pmatrix}$ &
		$\begin{pmatrix}
			&
		\end{pmatrix}$ &
		Сообщение об ошибке \\ \hline

		$\begin{pmatrix}
			1 & 2 & 3
		\end{pmatrix}$ &
		$\begin{pmatrix}
			1 & 2 & 3
		\end{pmatrix}$ &
		Сообщение об ошибке \\ \hline

		$\begin{pmatrix}
			1 \\
			1 \\
			1
		\end{pmatrix}$ &
		$\begin{pmatrix}
			1 \\
			1 \\
			1
		\end{pmatrix}$ &
		Сообщение об ошибке \\ \hline

		$\begin{pmatrix}
			1 & a & 3
		\end{pmatrix}$ &
		$\begin{pmatrix}
			1 \\
			1 \\
			1
		\end{pmatrix}$ &
		Сообщение об ошибке \\ \hline
			
		$\begin{pmatrix}
			1 & 2 & 3
		\end{pmatrix}$ &
		$\begin{pmatrix}
			1 \\
			a \\
			1
		\end{pmatrix}$ &
		Сообщение об ошибке \\ \hline

		$\begin{pmatrix}
			1 & 1 & 1
		\end{pmatrix}$ &
		$\begin{pmatrix}
			1 \\
			1 \\
			1
		\end{pmatrix}$ &
		$\begin{pmatrix}
			1 & 1 & 1\\
			1 & 1 & 1 \\
			1 & 1 & 1
		\end{pmatrix}$ \\ \hline

		$\begin{pmatrix}
			1 & 2 & 3 \\
			4 & 5 & 6 \\
			7 & 8 & 9
		\end{pmatrix}$ &
		$\begin{pmatrix}
			1 & 0 & 0 \\
			0 & 1 & 0 \\
			0 & 0 & 1
		\end{pmatrix}$ &
		$\begin{pmatrix}
			1 & 2 & 3 \\
			4 & 5 & 6 \\
			7 & 8 & 9
		\end{pmatrix}$ \\ \hline

		$\begin{pmatrix}
			1 & 2 & 3 \\
			4 & 5 & 6 \\
			7 & 8 & 9
		\end{pmatrix}$ &
		$\begin{pmatrix}
			1 & 1 \\
			1 & 1 \\
			1 & 1 
		\end{pmatrix}$ &
		$\begin{pmatrix}
			6 & 6 \\
			15 & 15 \\
			24 & 24
		\end{pmatrix}$ \\ \hline

		$\begin{pmatrix}
			2
		\end{pmatrix}$ &
		$\begin{pmatrix}
			2
		\end{pmatrix}$ &
		$\begin{pmatrix}
			4
		\end{pmatrix}$ \\ \hline

		\end{tabular}
	\end{center}
\end{table}


\newpage
\section*{Вывод}

Сформированы критерии для функции перемножения матриц. Разработаны 3 алгоритма умножения с использованием языка программирования C++. Разработанные алгоритмы удовлтворяют сформированным требованиям.
