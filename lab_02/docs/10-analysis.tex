\chapter{Аналитическая часть}

В этом, а также во всех последующих разделах матрицей размера $m \cdot n$ будем называть прямоугольную таблицу., содержащую $m$ строк и $n$ столбуов.

Для начала дадим определение умножения матриц. Пусть есть матрица $A$ размера  $m \cdot n$ и матрица $B$ размера $n \cdot l$. Тогда матрицей $С = A \cdot B$ будем называть матрицу 
\begin{equation}
C = \begin{pmatrix}
	a_{11} & a_{12} & \ldots & a_{1l}\\
	a_{21} & a_{22} & \ldots & a_{2l}\\
	\vdots & \vdots & \ddots & \vdots\\
	a_{m1} & a_{m2} & \ldots & a_{ml}
\end{pmatrix}
\end{equation}
где

\begin{equation}
  \label{eq:mult}
  c_{ij} = \sum_{r=1}^{m} a_{ir}b_{rj} \quad (i=\overline{1,m}; j=\overline{1,l})
\end{equation}

\section{Стандартный алгоритм}


Стандартный алгоритм заключается в подсчете  значения для каждого $c_{ij}$ согласно формуле 1.2.


\section{Алгоритм Винограда}

Заметим, что значение каждого элемента $c_{ij}$ равно скалярному произведению вектора-строки 
\begin{equation}
	V = (v_1, v_2, \ldots, v_n)
\end{equation}
на вектор-столбец
\begin{equation}
	W = (w_1, w_2, \ldots, w_n)
\end{equation}
Таким образом получим, что:
\begin{equation}
	c_{ij} = V \cdot W = \sum_{i=0}^{n} v_iw_i
\end{equation}
Иначе это выражение можно записать как:
\begin{equation}
	\label{eq:sum_1}
	c_{ij} = V \cdot W = \sum_{i=0, 2}^{n} ((v_i + w_i)(v_{i+1} + w_{i+1})) - \sum_{i=0, 2}^{n} (v_{i}v_{i+1} + w_{i}w_{i+1})
\end{equation}

Также заметим, что второе слагаемое выражения \ref{eq:sum_1} можно превычислить заранее для каджой строки левой матрицы и для каждого столбца правой.

Из-за такого подхода алгоритм Винограда в среднем будет использовать меньше умножений, чем стандартный алгоритм.

\section{Алгоритм Винограда с оптимизациями}

Согласно варианту необходимо работу алгоритма со следующими оптимизациями:
\begin{itemize}
	\item заменить $=$ на $+=$ там, где это возможно;
	\item предвычислять некоторые выражения;
	\item использовать битовый сдвиг вместо умножения на 2.
\end{itemize}



\section*{Вывод}

Рассмотрены теоретические аспекты алгоритмов умножения матриц. Получены достаточные знания для программной реализации.

\clearpage