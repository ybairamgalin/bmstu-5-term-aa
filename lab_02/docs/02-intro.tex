\chapter*{Введение}
\addcontentsline{toc}{chapter}{Введение}

Матрицей размера $m \cdot n$ называют прямоугольную таблицу из элементов произвольного типа, имеющую $m$ строк и $n$ столбцов.


Целью работы является сравнение трудоемкости различных алгоритмов перемножения матриц, а именно обычного алгоритма, алгоритма Винограда и алгоритма винограда с оптимизациями, согласно варианту.

Для достижения цели ставятся следующие задачи:
\begin{itemize}
	\item изучить классический алгоритм умножения матриц, алгоритм Винограда и модифицированный алгоритм Винограда;
	\item реализовать классический алгоритм умножения матриц, алгоритм
	Винограда и оптимизированный алгоритм Винограда;
	\item дать оценку трудоёмкости алгоритмов;
	\item замерить время работы алгоритмов;
	\item провести сравнительный анализ на основе полученных экспериментально данных.
\end{itemize}
