\chapter{Технологическая часть}

В данном разделе будут приведены требования к программному обеспечению, используемые технологии и реализации алгоритмов.

\section{Требования к программному обеспечению}

К программе предъявляется ряд требований:

\begin{itemize}
	\item запрос поступает по заданному адресу сетевого соединения на порт 5002;
	\item запрос поступает в формате http запроса, версия протокола 1.1;
	\item в запросе HTTP метод — POST;
	\item в запросе путь к ресурсу — /api/multiply\_matrix;
	\item в запросе указан заголовок Content-Type, его значение — text;
	\item в запросе указан заголовок Content-Length, его значение — длина тела запроса;
	\item в теле запроса располагаются через пробел (также допустимо использовать символ переноса строки вместо пробела) целые числа следующим образом: первые 2 числа $m_1$,  $n_1$ — размеры первой матрицы, далее расположены $m_1 \cdot n_1$ чисел — значения первой матрицы, далее 2 числа $m_2$, $n_2$ — размеры второй матрицы, далее $m_2 \cdot n_2$ чисел — значения второй матрицы;
	\item ответ поступает в формате http ответа.
	\item ответ содержит зоголовок Content-Type, его значение — text;
	\item ответ содержит заголовок Content-Length, его значение — длина тела запроса;
	\item при успешной обработке запроса в ответе содержится код выполнения 200, статус OK, в теле ответа через пробел (или знак переноса строки) располагаются целые числа, два первых из которых показывают размер полученной матрицы, остальные - содержание полученной матрицы;
	\item при ошибке соответствия формата запроса протоколу http запроса в ответе содержится код выполнения 400, статус Bad Request;
	\item при невозмомти умножения указанных в запросе матриц или несоответствия тела запроса формату, указанному выше в ответе содержится код выполнения 400, статус Bad Request.
\end{itemize}

\section{Выбор средств реализации}

В качестве языка программирования для реализации данной лабораторной работы был выбран язык программирования C++\cite{pythonlang}. Язык позволяет управлять всеми ресурсами компьютера и позволяет делать необхоимые системные вызовы.

Тестирование корректности работы серверного приложения производилось с помощью утилиты curl[2].

Метрики, изложенные в подразделе 2.6 замерены утилитой ab - Apache HTTP server benchmarking tool[3]

\section{Реализация алгоритмов}

Для параллельной обработки запросов были разработаны два класса: BlockingQueue, TreadPool. Их реализация представлена в лимтингах 3.1, 3.2 соответственно.

Непосредственно реализация серверного приложения представлена в классе Server в листинге 3.3. 

\begin{lstlisting}[caption=Реализация блокирующей очереди]
#pragma once

#include <deque>
#include <mutex>
#include <thread>
#include <chrono>
#include <condition_variable>

template<typename Task>
class BlockingQueue {
	public:
	void Add(Task task) {
		std::lock_guard guard(mutex_);
		queue_.push_back(std::move(task));
		queue_not_empty_.notify_one();
	}
	
	Task Take() {
		std::unique_lock lock(mutex_);
		while (true) {
			if (!queue_.empty()) {
				return TakeLocked();
			} else {
				queue_not_empty_.wait(lock);
			}
		}
	}
	
	private:
	Task TakeLocked() {
		Task oldest = std::move(queue_.front());
		queue_.pop_front();
		return oldest;
	}
	
	std::deque<Task> queue_; // Guarded by mutex_
	std::mutex mutex_;
	std::condition_variable queue_not_empty_;
};

\end{lstlisting}

\begin{lstlisting}[caption=Реализация пула потоков]
#pragma once

#include <thread>
#include <functional>
#include <deque>
#include <vector>
#include <stdexcept>


#include "pool/blocking_queue.hpp"

template<typename Task>
class StaticThreadPool {
	public:
	explicit StaticThreadPool(int thread_count)
	: thread_count_(thread_count) {
		StartWorkers();
	}
	
	void Submit(Task task) {
		blocking_queue_.Add(std::move(task));
	}
	
	void Join() {
		for (size_t i = 0; i < thread_count_; ++i) {
			blocking_queue_.Add({});
		}
	}
	
	private:
	void StartWorkers() {
		if (!pool_.empty()) {
			throw std::runtime_error("Bad start workers call");
		}
		
		for (int i = 0; i < thread_count_; ++i) {
			std::cout << "Starting worker " << i  << "\n";
			pool_.emplace_back([this]() {
				WorkerRoutine();
			});
		}
	}
	
	void WorkerRoutine() {
		while(true) {
			auto task = blocking_queue_.Take();
			if (!task) {
				break;
			}
			task();
		}
	}
	
	const int thread_count_;
	BlockingQueue<Task> blocking_queue_;
	std::vector<std::thread> pool_;
};

\end{lstlisting}

\begin{lstlisting}[caption=Реализация сервара из файла server/server.hpp]
#pragma once

#include <iostream>
#include <string>
#include <functional>

#include "pool/thread_pool.hpp"
#include "components.hpp"

class Server {
	using Task = std::function<void()>;
	public:
	Server();
	
	void StartListening();
	
	private:
	void ProcessRequest(int new_socket_descriptor) const;
	
	int socket_descriptor_;
	components::EndpointProvider endpoint_provider_;
	StaticThreadPool<Task> thread_pool_;
};

\end{lstlisting}

\begin{lstlisting}[caption=Реализация сервара из файла server/server.cpp]
#include "server.hpp"

#include <iostream>
#include <stdexcept>
#include <string>
#include <functional>
#include <memory>

#include <unistd.h>
#include <sys/socket.h>
#include <netinet/in.h>

#include "models/request.hpp"
#include "models/response.hpp"
#include "utils/parse.hpp"
#include "views/multiply_matrix/handler.hpp"
#include "views/handler.hpp"
#include "pool/thread_pool.hpp"
#include "components.hpp"

namespace {
	
	int kServerPortNumber = 5002;
	
}

Server::Server()
: socket_descriptor_(socket(AF_INET, SOCK_STREAM, 0)),
endpoint_provider_(components::EndpointProvider()),
thread_pool_(24) {
	endpoint_provider_.Register(
	views::Handler::HandleRequest,
	RequestType::kGet, "/");
	endpoint_provider_.Register(
	views::multiply_matrix::Handler::HandleRequest,
	RequestType::kPost, "/api/multiply_matrix");
	
	if (socket_descriptor_ == -1) {
		throw std::runtime_error("Socker error");
	}
	
	int enable = 1;
	const auto opt_rc = setsockopt(socket_descriptor_, SOL_SOCKET, SO_REUSEADDR,
	&enable, sizeof(int));
	if (opt_rc == -1) {
		throw std::runtime_error("Setopt error");
	}
	
	sockaddr_in address{};
	address.sin_family = AF_INET;
	address.sin_addr.s_addr = INADDR_ANY;
	address.sin_port = htons(kServerPortNumber);
	
	auto* converted_address = (sockaddr*)&address;
	if (bind(socket_descriptor_, converted_address, sizeof(address)) < 0) {
		throw std::runtime_error("Bind error");
	}
	
	std::cout << "Server is ready to accept connections\n";
}

void Server::StartListening() {
	listen(socket_descriptor_, 100);
	
	sockaddr_in client_address{};
	socklen_t client_length = sizeof(client_address);
	std::cout << "Listening on port " << kServerPortNumber << "\n";
	
	while (true) {
		const auto new_sfd = accept(
		socket_descriptor_, (sockaddr*)&client_address, &client_length);
		if (new_sfd < 0) {
			throw std::runtime_error("Accept error");
		}
		std::this_thread::sleep_for(std::chrono::milliseconds(1));
		thread_pool_.Submit([this, new_sfd]() {
			ProcessRequest(new_sfd);
			close(new_sfd);
		});
	}
}

void Server::ProcessRequest(int new_socket_descriptor) const {
	constexpr const size_t buffer_size = 524288;
	std::unique_ptr<char[]> buffer = std::make_unique<char[]>(buffer_size);
	bzero(buffer.get(), buffer_size);
	
	if (recv(new_socket_descriptor, buffer.get(),
	buffer_size, 0) == -1) {
		throw std::runtime_error("Read error");
	}
	
	Request request{};
	Response response{};
	try {
		request = utils::Parse(buffer.get(), utils::To<Request>());
		const auto endpoint = endpoint_provider_.Get(request.type, request.path);
		response = endpoint(request);
	} catch (std::exception& error) {
		std::cerr << "Error occurred while handling request : " << error.what();
		response = Response{500, "Server error"};
	}
	
	const auto response_string = BuildResponseString(response);
	const auto written_bytes = write(
	new_socket_descriptor, response_string.c_str(), response_string.size());
	
	if (written_bytes < 0) {
		std::cerr << "Write error";
	}
}	
\end{lstlisting}


\section{Тестрование}

Рассмотрены несколько тестовых случае, указанных ниже.

\begin{itemize}
	\item запрос из листинга 3.5 - ответ из листинга 3.6
	\begin{lstlisting}[caption={Запрос, проверяющий правильность ответа на корректных данных}]
	post /api/multiply_matrix HTTP/1.0
	Host: localhost:5002
	User-Agent: curl/7.84.0
	Accept: */*
	Content-Type: text
	Content-Length: 25
	
	2 2
	5 7
	9 1
	2 2
	7 4
	1 -5
	\end{lstlisting}
\end{itemize}

\begin{itemize}
	\item запрос из листинга 3.5 - ответ из листинга 3.6
	\begin{lstlisting}[caption={Ожидаемый ответ на запрос из листинга 3.5}]
	HTTP/1.1 200 OK
	Content-Length: 25
	Content-Type: text
	
	2 2
	42 -15
	64 31
	\end{lstlisting}
\end{itemize}

\newpage
\section*{Вывод}

Разработано серверное приожение, удовлетворяющее изложенным требованиям
