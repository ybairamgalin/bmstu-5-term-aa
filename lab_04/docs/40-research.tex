\chapter{Исследовательская часть}

В данном разделе будут приведены примеры работы программ,
постановка эксперимента и сравнительный анализ алгоритмов на основе
полученных данных.

\section{Технические характеристики}

Технические характеристики устройства, на котором выполнялось тестирование,
следующие.

\begin{itemize}
	\item Операционная система: macOs Monterey 12.4\cite{ubuntu}.
	\item Память: 16 Гбайт.
	\item Процессор: 2,6 ГГц 6‑ядерный процессор Intel Core i7\cite{intel}.
\end{itemize}

Тестирование проводилось на ноутбуке, включенном в сеть электропитания.
Во время тестирования ноутбук был нагружен только встроенными приложениями
окружения, а также непосредственно системой тестирования.


\section{Анализ метрик}

В таблице 4.1 представлена зависимость времени выполнения запросов в персентелях от количества потоков в системе.

В таблице 4.2 представлено количетсов запросов в секунду, которое сервер может успешно обработать.


\begin{table}[h]
	\begin{center}
		\captionsetup{justification=raggedright,singlelinecheck=off}
		\caption{\label{tab:time0}Количество запросов в секунду, которое может обработать сервер}
		\begin{tabular}{|c|c|c|c|c|c|c|}
			\hline
			Потоков, ед. & 50\%, мкс & 90\%, мкс & 95\%, мкс & 98\%, мкс & 100\%, мкс\\
			\hline
			2  & 3956 & 4011 & 4162 & 4254 & 4287 \\
			\hline
			6  & 904 & 947 & 964 & 1002 & 1075 \\
			\hline
			12  & 734 & 792 & 901 & 932 & 1154 \\
			\hline
			23  & 710 & 790 & 810 & 854 & 859 \\
			\hline
			31  & 726 & 885 & 900 & 1023 & 1119 \\
			\hline
		\end{tabular}
	\end{center}
\end{table}

\begin{table}[h]
	\begin{center}
		\captionsetup{justification=raggedright,singlelinecheck=off}
		\caption{\label{tab:time0}Количество запросов в секунду, которое может обработать сервер}
		\begin{tabular}{|c|c|}
			\hline
			Потоков, ед. & rps\\
			\hline
			2  & 20 \\
			\hline
			6  &32 \\
			\hline
			12  & 111 \\
			\hline
			23  & 198 \\
			\hline
			31  & 131 \\
			\hline
		\end{tabular}
	\end{center}
\end{table}

\pagebreak
\section*{Вывод}

Оптимальным является количествтво потоков, соответствующее количеству виртуальных ядер.
