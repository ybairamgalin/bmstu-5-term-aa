\chapter*{Введение}
\addcontentsline{toc}{chapter}{Введение}

Ранее на всех предметах в МГТУ им. Баумана рассматривались лишь алгоритмы,
которые выполняются последовательно. В данной лабораторной работе предстоит
погрузится в безграничный мир параллельных вычислений.

Многопоточность в современных информационных системах используется повсеместно,
так как во многих случаях позволяет обрабатывать большие массивы данных
за то же  процессорное время.

Целью данной лабораторной работы являтеся создание серверного приложения, способного эффективно обрабатывать множество запросов параллельно, нахождение оптимального количества потоков для получения наилучшего результата.

Для достижения цели ставятся следующие задачи:
\begin{itemize}
	\item разработать простое серверное приложение;
	\item распараллелить обработку поступающих запросов;
	\item сравнить время обработки запросов и максимальное количество запросов в секунду, которое способно обработать серверное приложение;
	\item оценить оптимальное количество обслуживающих потоков;
	\item провести сравнительный анализ на основе полученных экспериментально данных.
\end{itemize}
