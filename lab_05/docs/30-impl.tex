\chapter{Технологическая часть}

В данном разделе будут приведены требования к программному обеспечению, используемые технологии и реализации алгоритмов.

\section{Требования к программному обеспечению}

К программе предъявляется ряд требований:

\begin{itemize}
	\item запрос поступает по заданному адресу сетевого соединения на порт 5002;
	\item запрос поступает в формате http запроса, версия протокола 1.1;
	\item в запросе HTTP метод — POST;
	\item в запросе путь к ресурсу — /api/multiply\_matrix;
	\item в запросе указан заголовок Content-Type, его значение — text;
	\item в запросе указан заголовок Content-Length, его значение — длина тела запроса;
	\item в теле запроса располагается единственное положительное целое число - количество записей, которые требуется вернуть;
	\item ответ поступает в формате http ответа.
	\item ответ содержит зоголовок Content-Type, его значение — text;
	\item ответ содержит заголовок Content-Length, его значение — длина тела запроса;
	\item при успешной обработке запроса в ответе содержится код выполнения 200, статус OK, в теле ответа массив автомобильных номеров из базы данных;
	\item при ошибке соответствия формата запроса протоколу http запроса в ответе содержится код выполнения 400, статус Bad Request;
\end{itemize}

\section{Выбор средств реализации}

В качестве языка программирования для реализации данной лабораторной работы был выбран язык программирования C++\cite{pythonlang}. Язык позволяет управлять всеми ресурсами компьютера и позволяет делать необхоимые системные вызовы.

Тестирование корректности работы серверного приложения производилось с помощью утилиты curl[2].

Метрики, изложенные в подразделе 2.6 замерены утилитой ab - Apache HTTP server benchmarking tool[3]

\section{Реализация алгоритмов}

Для параллельной обработки запросов были разработаны два класса: BlockingQueue, TreadPool. Их реализация представлена в лимтингах 3.1, 3.2 соответственно.

Непосредственно реализация серверного приложения представлена в классе Server в листинге 3.3. 

\begin{lstlisting}[caption=Реализация блокирующей очереди]
#pragma once

#include <deque>
#include <mutex>
#include <thread>
#include <chrono>
#include <condition_variable>

template<typename Task>
class BlockingQueue {
	public:
	void Add(Task task) {
		std::lock_guard guard(mutex_);
		queue_.push_back(std::move(task));
		queue_not_empty_.notify_one();
	}
	
	Task Take() {
		std::unique_lock lock(mutex_);
		while (true) {
			if (!queue_.empty()) {
				return TakeLocked();
			} else {
				queue_not_empty_.wait(lock);
			}
		}
	}
	
	private:
	Task TakeLocked() {
		Task oldest = std::move(queue_.front());
		queue_.pop_front();
		return oldest;
	}
	
	std::deque<Task> queue_; // Guarded by mutex_
	std::mutex mutex_;
	std::condition_variable queue_not_empty_;
};

\end{lstlisting}

\begin{lstlisting}[caption=Реализация пула потоков]
#pragma once

#include <thread>
#include <functional>
#include <deque>
#include <vector>
#include <stdexcept>


#include "pool/blocking_queue.hpp"

template<typename Task>
class StaticThreadPool {
	public:
	explicit StaticThreadPool(int thread_count)
	: thread_count_(thread_count) {
		StartWorkers();
	}
	
	void Submit(Task task) {
		blocking_queue_.Add(std::move(task));
	}
	
	void Join() {
		for (size_t i = 0; i < thread_count_; ++i) {
			blocking_queue_.Add({});
		}
	}
	
	private:
	void StartWorkers() {
		if (!pool_.empty()) {
			throw std::runtime_error("Bad start workers call");
		}
		
		for (int i = 0; i < thread_count_; ++i) {
			std::cout << "Starting worker " << i  << "\n";
			pool_.emplace_back([this]() {
				WorkerRoutine();
			});
		}
	}
	
	void WorkerRoutine() {
		while(true) {
			auto task = blocking_queue_.Take();
			if (!task) {
				break;
			}
			task();
		}
	}
	
	const int thread_count_;
	BlockingQueue<Task> blocking_queue_;
	std::vector<std::thread> pool_;
};

\end{lstlisting}

\begin{lstlisting}[caption=Реализация сервара из файла server/server.hpp]
#pragma once

#include <iostream>
#include <string>
#include <functional>

#include "pool/thread_pool.hpp"
#include "components.hpp"

class Server {
	using Task = std::function<void()>;
	public:
	Server();
	
	void StartListening();
	
	private:
	int socket_descriptor_;
};

\end{lstlisting}

\begin{lstlisting}[caption=Реализация сервара и конвеера из файла server/server.cpp]
#include "server.hpp"

#include <iostream>
#include <stdexcept>
#include <string>
#include <functional>
#include <memory>

#include <unistd.h>
#include <sys/socket.h>
#include <netinet/in.h>

#include <pqxx/pqxx>

#include "models/request.hpp"
#include "models/response.hpp"
#include "utils/parse.hpp"
#include "views/handler.hpp"
#include "pool/thread_pool.hpp"
#include "components.hpp"

namespace {
	
	constexpr const int kServerPortNumber = 5002;
	
	constexpr const int kParseThreadCount = 2;
	constexpr const int kDbThreadCount = 8;
	constexpr const int kResultThreadCount = 2;
	
	inline constexpr const char kConnectionString[] =
	"host=localhost dbname=lab_01";
	
	using Task = std::function<void()>;
	
	StaticThreadPool<Task> thread_pool_parse{kParseThreadCount};
	StaticThreadPool<Task> thread_pool_db_connect{kDbThreadCount};
	StaticThreadPool<Task> thread_pool_response{kResultThreadCount};
	
	void ProcessResult(
	int socket_descriptor,
	const std::vector<std::string>& car_numbers) {
		Response response = [&](){
			if (car_numbers.empty()) {
				return Response{400, "Bas request"};
			}
			std::string response_string = "{\n";
				for (const auto& number : car_numbers) {
					response_string += fmt::format("\t'{}',\n", number);
				}
				response_string += "}\n";
			
			return Response{200, response_string};
		}();
		
		const auto raw_response = BuildResponseString(response);
		write(socket_descriptor, raw_response.c_str(), raw_response.size());
	}
	
	void ProcessDb(
	int socker_descriptor,
	Request request) {
		if (!request.body.has_value()) {
			thread_pool_response.Submit([=](){
				ProcessResult(socker_descriptor, {});
			});
		}
		
		int count_from_requests;
		try {
			count_from_requests = std::stoi(request.body.value());
		} catch (const std::exception& e) {
			thread_pool_response.Submit([=](){
				ProcessResult(socker_descriptor, {});
			});
			return;
		}
		
		std::vector<std::string> result;
		const auto query = fmt::format(
		"select state_number "
		"from public.cars "
		"order by state_number "
		"limit {}", count_from_requests);
		try {
			pqxx::connection db_connection(kConnectionString);
			pqxx::work transaction{db_connection};
			for (auto [number] : transaction.stream<std::string>(query)) {
				result.push_back(number);
			}
		} catch (const std::exception& e) {
			std::cout << e.what();
			return;
		}
		
		thread_pool_response.Submit([=](){
			ProcessResult(socker_descriptor, result);
		});
	}
	
	void ProcessParse(int socket_descriptor) {
		constexpr const size_t buffer_size = 524288;
		std::unique_ptr<char[]> buffer = std::make_unique<char[]>(buffer_size);
		bzero(buffer.get(), buffer_size);
		
		if (recv(socket_descriptor, buffer.get(),
		buffer_size, 0) == -1) {
			throw std::runtime_error("Read error");
		}
		
		Request request{};
		try {
			std::cout << buffer.get();
			request = utils::Parse(buffer.get(), utils::To<Request>());
		} catch (std::exception& error) {
			std::cerr << "Error occurred while handling request : " << error.what();
		}
		
		thread_pool_db_connect.Submit([=](){
			ProcessDb(socket_descriptor, request);
		});
	}
	
}

Server::Server()
: socket_descriptor_(socket(AF_INET, SOCK_STREAM, 0)) {
	if (socket_descriptor_ == -1) {
		throw std::runtime_error("Socker error");
	}
	
	int enable = 1;
	const auto opt_rc = setsockopt(socket_descriptor_, SOL_SOCKET, SO_REUSEADDR,
	&enable, sizeof(int));
	if (opt_rc == -1) {
		throw std::runtime_error("Setopt error");
	}
	
	sockaddr_in address{};
	address.sin_family = AF_INET;
	address.sin_addr.s_addr = INADDR_ANY;
	address.sin_port = htons(kServerPortNumber);
	
	auto* converted_address = (sockaddr*)&address;
	if (bind(socket_descriptor_, converted_address, sizeof(address)) < 0) {
		throw std::runtime_error("Bind error");
	}
	
	std::cout << "Server is ready to accept connections\n";
}

void Server::StartListening() {
	listen(socket_descriptor_, 100);
	
	sockaddr_in client_address{};
	socklen_t client_length = sizeof(client_address);
	std::cout << "Listening on port " << kServerPortNumber << "\n";
	
	while (true) {
		const auto new_sfd = accept(
		socket_descriptor_, (sockaddr*)&client_address, &client_length);
		if (new_sfd < 0) {
			throw std::runtime_error("Accept error");
		}
		std::this_thread::sleep_for(std::chrono::milliseconds(1));
		thread_pool_parse.Submit([new_sfd]() {
			::ProcessParse(new_sfd);
		});
	}
}

\end{lstlisting}


\section{Тестрование}

Рассмотрены несколько тестовых случае, указанных ниже.

\begin{itemize}
	\item запрос из листинга 3.5 - ответ из листинга 3.6
	\begin{lstlisting}[caption={Запрос, проверяющий правильность ответа на корректных данных}]
	post /api/multiply_matrix HTTP/1.0
	Host: localhost:5002
	User-Agent: curl/7.84.0
	Accept: */*
	Content-Type: text
	Content-Length: 25
	
	2 2
	5 7
	9 1
	2 2
	7 4
	1 -5
	\end{lstlisting}
\end{itemize}

\begin{itemize}
	\item запрос из листинга 3.5 - ответ из листинга 3.6
	\begin{lstlisting}[caption={Ожидаемый ответ на запрос из листинга 3.5}]
	HTTP/1.1 200 OK
	Content-Length: 25
	Content-Type: text
	
	2 2
	42 -15
	64 31
	\end{lstlisting}
\end{itemize}

\newpage
\section*{Вывод}

Разработано серверное приожение, удовлетворяющее изложенным требованиям
